\section{Data Scarcity and Noisy Datasets in Machine Learning}

\textbf{Machine Learning} (ML) is reshaping our world in ways both
subtle and profound. From the personalized recommendations we receive
on streaming platforms to the autonomous vehicles navigating in real streets,
ML algorithms are becoming an integral part of our daily lives.
Yet, as these systems tackle increasingly complex challenges,
they reveal their limitations.
One of the most pressing issues is the critical need for large, 
high-quality datasets, a requirement that is often difficult to
meet in real-world scenarios.

The \textbf{scarcity} of comprehensive datasets poses a significant
obstacle in many fields, including healthcare and robotics.
In \textbf{healthcare}, privacy concerns and limited access to patient
records hinder the collection of diverse and extensive data.
Similarly, in \textbf{robotics}, certain experimental data may be difficult
or costly to obtain, particularly in scenarios involving complex
physical interactions or rare events. This data scarcity can lead
to underfitted or non-generalizable models, compromising their
accuracy and utility in practical applications.

Moreover, even when substantial datasets are available, they
often suffer from \textbf{noise and inconsistencies}. Real-world data
collection processes are rarely perfect, resulting in datasets
that may contain errors, outliers, or biased samples.
These imperfections can significantly impact the performance of ML models,
leading to unreliable predictions or decisions. The challenge of dealing
with noisy datasets is particularly acute in offline
\textbf{Deep Reinforcement Learning} (DRL) scenarios, where agents must learn
optimal policies from pre-collected experiences without the ability
to interact further with the environment.

To address these challenges, researchers have long employed
\textbf{data augmentation techniques}. Traditional data augmentation
methods involve
applying various transformations to existing data points to artificially
expand the dataset. In image processing, for instance, this might involve
rotations, flips, or color adjustments. While these techniques have proven
effective in many applications, they have limitations.
Notably, not all transformations produce valid or realistic data points,
especially in domains with complex underlying structures or causal relationships.

This is where the integration of \textbf{Causal Inference} into
ML frameworks offers a
promising solution. Causal inference focuses on understanding the
cause-and-effect relationships between variables, rather
than merely identifying correlations. By leveraging causal inference,
we can develop more sophisticated data augmentation techniques that
respect the underlying causal structure of the data-generating process.

In this thesis, we propose novel approaches to address the
challenges of data scarcity and noisy datasets by leveraging
causal inference techniques. We introduce two frameworks for
counterfactual data generation:
the \textbf{Wasserstein Reward-enhanced Counterfactual Data Generation}
(WRe-CTDG) and the \textbf{End-to-End Counterfactual Data Generation} (e2e-CTDG).
These methods aim to augment pre-collected datasets by generating additional
high-fidelity experiences that align with the environment's underlying
transition dynamics.

Our work focuses specifically on offline DRL scenarios, where agents
are trained using pre-collected datasets of experiences.
We explore applications in both robotics environments
(such as Acrobot, Half Cheetah, and Pusher) and healthcare scenarios
(such as diabetes treatment simulation).
These domains present unique challenges due to their complex
dynamics and the potential consequences of incorrect decisions.

By incorporating \textbf{Structural Causal Models} (SCMs) and
\textbf{Generative Adversarial Networks} (GANs) into our frameworks,
we develop techniques to infer causal relationships and generate counterfactual
samples. This approach allows us to create synthetic data points
that are not mere transformations of existing data, but rather plausible
alternative scenarios that could have occurred under different circumstances.
This is particularly valuable in offline DRL, where the ability to explore
``what-if'' scenarios can significantly enhance the robustness
and generalization capabilities of learned policies.

\section{Thesis Overview}

The following chapters are organized as follows:
\begin{itemize}
    \item In \textbf{Chapter 2}, we provide an overview of the State of the Art
    in Offline Reinforcement Learning, Data Augmentation and Generation Techniques,
    and Counterfactual Reasoning in Data Generation.
    \item In \textbf{Chapter 3}, we introduce the theoretical foundations
    of the tecnologies used in our proposed frameworks,
    including Structural Causal Models,
    Generative Adversarial Networks,
    and Deep Reinforcement Learning algorithms.
    \item In \textbf{Chapter 4}, we present the WRe-CTDG
    and the e2e-CTDG frameworks,
    detailing their architectures and training procedures.
    We also present the environments and datasets used for evaluation.
    \item In \textbf{Chapter 5}, we present a comprehensive description
    of the machines and neural networks employed in our experiments,
    followed by a detailed overview of the experimental results.
    \item In \textbf{Chapter 6}, we discuss the implications of our findings,
    the limitations of our approaches,
    and potential directions for future research.
\end{itemize}